
\documentclass{acm_proc_article-sp}

\begin{document}

\title{Conversing verses - haiku generation using a LSTM-based auto-encoder matching model}

\numberofauthors{2} 
\author{
\alignauthor
Luka Ernestini\\
       \affaddr{Univerza v Mariboru}\\
       \affaddr{Fakulteta za elektrotehniko, računalništvo in informatiko}\\
       \affaddr{Maribor, Slovenija}\\
       \email{luka.ernestini@student.um.si}

\alignauthor
Niko Uremović\\
       \affaddr{Univerza v Mariboru}\\
       \affaddr{Fakulteta za elektrotehniko, računalništvo in informatiko}\\
       \affaddr{Maribor, Slovenija}\\
       \email{niko.uremovic@um.si}
}

\maketitle
\begin{abstract}

Lorem ipsum

\end{abstract}

% todo: kaj to pomeni
\category{H.4}{Information Systems Applications}{Miscellaneous}

% todo: kaj to pomeni
\terms{Theory}

\keywords{text generation, neural networks, LSTM}

\section{Introduction}

- neural networks so zakon
       - RNN, LSTM - generiranje zgodbe \cite{pawade2018story}
       - RNN, LSTM - abstracitve text summarization \cite{song2019abstractive}
       - CNN za image classification, RNN (LSTM) za caption generation - image captioning \cite{You_2016_CVPR}
       - sklepanje, generiranje hipotez na podlagi naravnega besedila  \cite{bhagavatula2020abductive}
tezave s katerimi se metode srecujejo (objektivno, avtomatizirano ovrednotenje - semanticna smiselnost besedila)
       - ponekod lahko uporabimo metrike (kjer imamo target/label text) - image captioning
              - BLEU \cite{papineni2002bleu}
              - METEOR \cite{banerjee2005meteor}
       - ponekod nimamo "pravega"/target/label teksta, in tako niso mozna avtomatska ovretnotenja (ovrednotijo anketiranci, human evaluation)
              - generiranje kitajskih pesmi \cite{zhang2014chinese}
              - Haiku generation using seed word(s), associations, heuristics for selecting optimal one \cite{netzer2009gaiku}.

- clanki s katerimi primerjamo, vzamemo za osnovo
       - generiranje vsakdanjih pogovorov. Tri nevronske mreze se uporabijo za; LSTM za encodanje ene replike v semanticno predstavitev -> simple feedforward mreza za mapiranje semantike replike v semantiko odgovora -> LSTM decoder za tvorjenje stavka iz dobljene nove semantike. \cite{luo2018autoencoder}
       - generiranje besedila rapperskih pesmi z LSTM \cite{potash2015ghostwriter}.
       - primerjamo lahko svojo metodo z \cite{netzer2009gaiku}, ki tudi se ukvarja z generiranjem haikujev

- ideja - generiranje haikov, nova metoda
motivacija
       - LSTM ima zmoznost posnemati ritem, znacilnosti poezije, kot so uporaba za zvrst znacilnih fraz, posnemanje sloga pisanja \cite{potash2015ghostwriter}
       - nov nacin navezave med verzi - pogovor (vrstici haikuja, encoder-decoder se pogovarjata)
metodologija, postopek
       - generiranje prve vrstice z lstm
       - generiranje naslednjih vrstic v obliki pogovora (uporaba encoder/decoder pristopa) % implementacija https://github.com/lancopku/AMM

\section{Drugi naslovi}

Lorem ipsum

\section{Conclusions}

Lorem ipsum

\section{Acknowledgments}

The authors acknowledge lorem ipsum.

\bibliographystyle{abbrv}
\bibliography{sigproc}


\balancecolumns

\end{document}
